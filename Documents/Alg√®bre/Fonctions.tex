\documentclass[a4paper,11pt]{article}
\usepackage{commonpackages}

\begin{document}
\title{Fonctions du premier degré}
\date{}
\maketitle

\section{Théorie}
Les fonctions qui peuvent s'écrire sous la forme
$$f(x)=a\cdot x +b$$
où $a$ et $b$ sont des nombres réels, sont appelées \textbf{fonctions du premier degré}.

Les graphes des fonctions du premier degré sont \textbf{toujours des droites}.

\section{Exercice}
Sur le graphique ci-dessous, tu peux observer comment change la droite en fonction de $a$ qui représente la pente et $b$ qui est l'ordonnée à l'origine:\par
\geogebra{esdhdhzd}
\begin{enumerate}
    \item Comment est la droite quand $a = 3$?
    \item Comment est la droite quand $a = -4$?
    \item Comment est la droite quand $a = 0$?
    \item Que se passe-t-il quand la valeur de $b$ change?
\end{enumerate}

\begin{solution}
\begin{enumerate}
    \item La droite est croissante (elle monte).
    \item La droite est décroissante (elle descend).
    \item La droite est constante (elle est horizontale/parallèle à l'axe de x)
    \item La droite se déplace vers le haut ou vers le bas, mais l'inclinaison reste la même.
\end{enumerate}
\end{solution}

\section{Théorie}
Les fonctions qui peuvent s'écrire sous la forme
$$f(x)=a\cdot x^2 + b \cdot x + c$$
où $a$, $b$ et $c$ sont des nombres réels et $a \ne 0$, sont appelées \textbf{fonctions quadratiques} ou \textbf{fonctions du 2\textsuperscript{ème} degré}.

Les graphes des fonctions du deuxième degré sont \textbf{toujours des paraboles}.

\section{Exercice}
Sur le graphique ci-dessous, tu peux observer comment change la parabole en fonction de $a$, $b$ et $c$:\par
\geogebra{jjenaqny}
\begin{enumerate}
    \item Comment est la parabole quand $a$ est positif?
    \item Comment est la parabole quand $a$ est négatif?
    \item Que fait la parabole si $a$ augmente ou diminue sans changer de signe?
    \item Que fait la parabole quand $c$ change?
    \item Que fait la parabole quand $b$ change?
\end{enumerate}

\begin{solution}
\begin{enumerate}
    \item La parabole "sourit". Elle est convexe.
    \item La parabole "fait la tête". Elle est concave.
    \item L'ouverture de la parabole change.
    \item La parabole se déplace verticalement.
    \item La parabole se déplace.
\end{enumerate}
\end{solution}

\section{Définition}
Une relation $f: A \longrightarrow B$, qui associe à chaque élément de l'ensemble de départ $A$ au plus un élément de l'ensemble d'arrivée $B$, est appelée \textbf{fonction}.
Si $x$ est un élément de $A$, alors $f(x)$, si elle existe, est unique et est appelée l'\textbf{image} de $x$ par $f$.
Inversement $x$ est appelé la \textbf{préimage} ou l'\textbf{antécédent} de $f(x)$.
Une valeur $f(x)$ peut avoir plusieurs préimages.

\section{Exercice}
Calculer $f(2)$, $f(-1)$ et $f(9)$ pour les fonctions suivantes.
\begin{multicols}{2}
\begin{enumerate}
\item $f(x)=4x-3$
\item $f(x)=x^2-3x+1$
\item $f(x)=-\dfrac{x}{4}+14$
\item $f(x)=\dfrac{1}{2}x^3-4x+2$
\end{enumerate}
\end{multicols}

\begin{solution}
\begin{enumerate}
\item $f(2)=5$; $f(-1)=-7$; $f(9)=33$
\item $f(2)=-1$; $f(-1)=5$; $f(9)=55$
\item $f(2)=\dfrac{27}{2}=13.5$; $f(-1)=\dfrac{57}{4}=14.25$; $f(9)=\dfrac{47}{4}=11.75$
\item $f(2)=-2$; $f(-1)=\dfrac{11}{2}=5.5$; $f(9)=\dfrac{661}{2}=330.5$
\end{enumerate}
\end{solution}

\end{document}
