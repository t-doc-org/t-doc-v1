\documentclass[a4paper,11pt]{article}
\usepackage{commonpackages}

\begin{document}
\title{Équations du premier degré}
\date{}
\maketitle

\section{théorie}
Une \textbf{équation du premier degré} est une égalité entre deux expressions littérales dont le coefficient de la variable est au maximum 1.
$$\underbrace{2x + 3}_{\text{membre de gauche}} = \underbrace{5 - 5x}_{\text{membre de droite}}$$
La ou les valeurs qui vérifient l'égalité sont appelées solutions de l'équation. Résoudre une équation, c'est trouver l'\textbf{ensemble des solutions}, noté $S$.

Pour résoudre une équation, on utilise les \textbf{règles d'équivalence}, celles-ci transforment l'équation sans modifier son ensemble de solutions. Les trois règles d'équivalence sont:
\begin{enumerate}[label*=\arabic*.]
\item effectuer du calcul littéral dans ses membres.
\item additionner (ou soustraire) un même nombre, un même monôme ou un même polynôme aux deux membres de l'équation.
\item multiplier (ou diviser) les deux membres de l'équation par un même nombre \textbf{non nul}.
\end{enumerate}

\section{exemple}
$\begin{array}{rclrll}
2x-5+x-2 &=&2x+8-4x &|& \text{CL} & \text{ (calcul littéral)} \\
3x-7 &=&-2x+8 &|& +2x & \text{ (addition du même monôme)} \\
5x-7 &=& 8 &|& +7 &  \text{ (addition du même nombre)}\\
5x &=& 15 &|& :5 & \text{ (division par le même nombre)}\\
x &=& 3 &&&\\
S &=& \{3\} &&& \text{ (noter l'ensemble des solutions)}
\end{array}$

\section{vidéo}
Théorie sur les équations du 1er degré et exemples:\par
% \video{WoTpA2RyuVU?si=pB7ds5xAY3ulD1I8} %indiquer le pourcentage voulu par rapport à la largeur de la page

\subsection{Représentation graphique}
Les fonctions qui peuvent s'écrire sous la forme
$$f(x)=a\cdot x +b$$
où $a$ et $b$ sont des nombres réels, sont appelées \textbf{fonctions du premier degré}.

Les graphes des fonctions du premier degré sont \textbf{toujours des droites}.

Sur le graphique ci-dessous, tu peux observer comment change la droite en fonction de a qui représente la pente et b qui est l'ordonnée à l'origine:\par
\geogebra[100]{esdhdhzd} %indiquer le pourcentage voulu par rapport à la largeur de la page
\begin{enumerate}
    \item Comment est la droite quand a = 3?
    \item Comment est la droite quand a = -4?
    \item Comment est la droite quand a = 0?
    \item Que se passe-t-il quand on change la valeur de b?
\end{enumerate}

\begin{solution}
\begin{enumerate}
    \item La droite est croissante (elle monte).
    \item La droite est décroissante (elle descend).
    \item La droite est constante (elle est horizontale/parallèle à l'axe de x)
    \item La droite se déplace vers le haut ou vers le bas, mais l'inclinaison reste la même.
\end{enumerate}
\end{solution}

\subsection{Définition}
Une relation $f: A \longrightarrow B$, qui associe à chaque élément de l'ensemble de départ $A$ au plus un élément de l'ensemble d'arrivée $B$, est appelée \textbf{fonction}.
Si $x$ est un élément de $A$, alors $f(x)$, si elle existe, est unique et est appelée l'\textbf{image} de $x$ par $f$.
Inversement $x$ est appelé la \textbf{préimage} ou l'\textbf{antécédent} de $f(x)$.
Une valeur $f(x)$ peut avoir plusieurs préimages.

\subsection{Exercice}
Calculer $f(2)$, $f(-1)$ et $f(9)$ pour les fonctions suivantes.
\begin{multicols}{2}
\begin{enumerate}
\item $f(x)=4x-3$
\item $f(x)=x^2-3x+1$
\item $f(x)=-\dfrac{x}{4}+14$
\item $f(x)=\dfrac{1}{2}x^3-4x+2$
\end{enumerate}
\end{multicols}

\begin{solution}
\begin{enumerate}
\item $f(2)=5$; $f(-1)=-7$; $f(9)=33$
\item $f(2)=-1$; $f(-1)=5$; $f(9)=55$
\item $f(2)=\dfrac{27}{2}=13.5$; $f(-1)=\dfrac{57}{4}=14.25$; $f(9)=\dfrac{47}{4}=11.75$
\item $f(2)=-2$; $f(-1)=\dfrac{11}{2}=5.5$; $f(9)=\dfrac{661}{2}=330.5$
\end{enumerate}
\end{solution}

\end{document}
