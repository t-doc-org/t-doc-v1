\documentclass[a4paper,11pt]{article}
\usepackage{commonpackages}

\begin{document}
\title{Équations du premier degré}
\date{}
\maketitle

\section{Théorie}
Une \textbf{équation du premier degré} est une égalité entre deux expressions littérales dont le coefficient de la variable est au maximum 1.
$$\underbrace{2x + 3}_{\text{membre de gauche}} = \underbrace{5 - 5x}_{\text{membre de droite}}$$
La ou les valeurs qui vérifient l'égalité sont appelées solutions de l'équation. Résoudre une équation, c'est trouver l'\textbf{ensemble des solutions}, noté $S$.

Pour résoudre une équation, on utilise les \textbf{règles d'équivalence}, celles-ci transforment l'équation sans modifier son ensemble de solutions. Les trois règles d'équivalence sont:
\begin{enumerate}[label*=\arabic*.]
\item effectuer du calcul littéral dans ses membres.
\item additionner (ou soustraire) un même nombre, un même monôme ou un même polynôme aux deux membres de l'équation.
\item multiplier (ou diviser) les deux membres de l'équation par un même nombre \textbf{non nul}.
\end{enumerate}

\section{Exemple}
$\begin{array}{rclrll}
2x-5+x-2 &=&2x+8-4x &|& \text{CL} & \text{ (calcul littéral)} \\
3x-7 &=&-2x+8 &|& +2x & \text{ (addition du même monôme)} \\
5x-7 &=& 8 &|& +7 &  \text{ (addition du même nombre)}\\
5x &=& 15 &|& :5 & \text{ (division par le même nombre)}\\
x &=& 3 &&&\\
S &=& \{3\} &&& \text{ (noter l'ensemble des solutions)}
\end{array}$

\section{Vidéo}
Théorie sur les équations du 1er degré et exemples:\par
\video{WoTpA2RyuVU?si=pB7ds5xAY3ulD1I8}

\end{document}
