\documentclass[a4paper,11pt]{article}
\usepackage{commonpackages}

\usepackage{xcolor}
\definecolor{mybackcolor}{rgb}{0.98,0.98,0.98}

\begin{document}
\title{Exercices avec Python}
\date{}
\maketitle

\section{Exercice}
Tu as reçu une punition et tu dois écrire "J'aime l'informatique" 100 fois.
Écrire un progamme qui permet de faire cela en une seule ligne.
\begin{solution}
\begin{verbatim}{python,interactive}
print("J'aime l'informatique! \n" * 100)
\end{verbatim}
\end{solution}


\section{Exercice}
Écrire un programme qui demande son prénom à l'utilisateur et qui répond:\\
Bonjour \textbf{...}\\
\begin{solution}
\begin{verbatim}{python,interactive}
nom = input("Quel est ton prénom? ")
print("Bonjour " + nom)
\end{verbatim}
\end{solution}

\section{Exercice}
Écrire un programme qui demande un nombre à l'utilisateur et qui retourne le carré de ce nombre:\\
\begin{solution}
\begin{verbatim}{python,interactive}
n = int(input("Choisir un nombre: "))
print(n**2)
\end{verbatim}
\end{solution}

\section{Exercice}
Écrire un programme qui son âge à l'utilisteur et retourne s'il est mineur ou majeur:\\
\begin{solution}
\begin{verbatim}{python,interactive}
age = int(input("Quel est votre âge? "))
if (age < 18):
  print("Tu es mineur.")
else:
  print("Tu es majeur.")
\end{verbatim}
\end{solution}

\section{Exercice}
Écrire un programme qui demande un nombre à l'utilisteur et écrit la table de multiplication de ce nombre:\\
\begin{solution}
\begin{verbatim}{python,interactive}
n = int(input("Choisir un nombre. "))
for i in range(1, 13):
  print(i * n)
\end{verbatim}
\end{solution}

\section{Exercice}
Test avec turtle:\\
\begin{solution}
\begin{verbatim}{python,interactive,turtle}
from turtle import *

for i in range(4):
    forward(20)
    penup()
    forward(20)
    pendown()
\end{verbatim}
\end{solution}

\begin{solution}
\begin{verbatim}{python,interactive,turtle}
from turtle import *

def carre100():
    forward(100)
    left(90)
    forward(100)
    left(90)
    forward(100)
    left(90)
    forward(100)
    left(90)

pensize(4)
carre100()
pencolor("red")
penup()
goto(0,200)
pendown()
carre100()
pencolor("green")
penup()
goto(-200,200)
pendown()
carre100()
pencolor("blue")
penup()
goto(-200,0)
pendown()
carre100()
\end{verbatim}
\end{solution}

\begin{solution}
\begin{verbatim}{python,interactive,turtle}
from turtle import *
from random import randint

nb_cote_max = int(input("Combien de côtés voulez-vous au maximum? "))
colormode(255)
pensize(3)
for i in range(3, nb_cote_max):
    for j in range(i):
        forward(100)
        left(360/i)
    pencolor(randint(0,255), randint(0, 255), randint(0,255))
\end{verbatim}
\end{solution}

\begin{solution}
\begin{verbatim}{python,interactive,turtle}
from turtle import *

nb_cote = int(input("Combien de côtés doit avoir le polygone? " ))

def polygone():
    for i in range(nb_cote):
        forward(100)
        left(360/nb_cote)

while nb_cote < 3:
    nb_cote = int(input("Cette valeur n'est pas valide. Combien de côtés doit avoir le polygone (une valeure supérieure ou égale à 3)? " ))

pensize(3)
polygone()
\end{verbatim}
\end{solution}

\begin{solution}
\begin{verbatim}{python,interactive,turtle}
from turtle import *

i = 1
speed(0)
while i <= 400:
    forward(i)
    left(90)
    i += 4
    print(i)
\end{verbatim}
\end{solution}





\end{document}
