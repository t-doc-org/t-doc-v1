\documentclass[a4paper,11pt]{article}
\usepackage{commonpackages}

\usepackage{xcolor}
\definecolor{mybackcolor}{rgb}{0.98,0.98,0.98}

\begin{document}
\title{Exercices avec Python - Console}
\date{}
\maketitle

\section{print() et input()}

\subsection{Exercice}
Écrire un programme qui affiche exactement ce texte:\\
Salut\\
Je suis élève au collège Sainte-Croix\\
J'ai 16 ans\\
J'aime bien jouer au volley\\
\begin{solution}
\begin{verbatim}{python,interactive}
print("Salut")
print("Je suis élève au collège Sainte-Croix")
print("J'ai 16 ans")
print("J'aime bien jouer au volley")
\end{verbatim}
\end{solution}

\subsection{Exercice}
Tu as reçu une punition et tu dois écrire "J'aime l'informatique" 100 fois.
Écrire un programme qui permet de faire cela en une seule ligne.
\begin{solution}
\begin{verbatim}{python,interactive}
print("J'aime l'informatique! \n" * 100)
\end{verbatim}
\end{solution}

\subsection{Exercice}
Écrire un programme qui demande son prénom à l'utilisateur et qui répond:\\
Bonjour \textbf{...}\\
\begin{solution}
\begin{verbatim}{python,interactive}
nom = input("Quel est ton prénom? ")
print("Bonjour " + nom)
\end{verbatim}
\end{solution}

\subsection{Exercice}
Écrire un programme qui demande un nombre à l'utilisateur et qui retourne le carré de ce nombre:\\
\begin{solution}
\begin{verbatim}{python,interactive}
n = int(input("Choisir un nombre: "))
print(n**2)
\end{verbatim}
\end{solution}

\section{Variables}

\subsection{Exercice}
Qu'affiche le programme suivant?
\begin{verbatim}{python,interactive}
a = 3
a = a - 1
print(a)
\end{verbatim}

\subsection{Exercice}
Qu'affiche le programme suivant?
\begin{verbatim}{python,interactive}
a = 7
b = a - 3
print(a*b)
\end{verbatim}

\subsection{Exercice}
Qu'affiche le programme suivant?
\begin{verbatim}{python,interactive}
a = 5
a += 2
a = a * a - 9
print(a)
\end{verbatim}

\subsection{Exercice}
Qu'affiche le programme suivant?
\begin{verbatim}{python,interactive}
a = 20
a += 1
print((a/3)**2)
\end{verbatim}

\subsection{Exercice}
Qu'affiche le programme suivant?
\begin{verbatim}{python,interactive}
a = 22
b = 5
print((a//b) + (a%b))
\end{verbatim}


\section{Instruction conditionnelle}

\subsection{Exercice}
Écrire un programme qui demande son âge à l'utilisteur et retourne s'il est mineur ou majeur:\\
\begin{solution}
\begin{verbatim}{python,interactive}
age = int(input("Quel est votre âge? "))
if (age < 18):
  print("Tu es mineur.")
else:
  print("Tu es majeur.")
\end{verbatim}
\end{solution}

\subsection{Exercice}
Que retourne le programme suivant?
\begin{verbatim}{python,interactive}
a = 2
if a != 2:
    print("a n'est pas égal à 2")
elif a > 2 :
    print("a est plus grand que 2")
else:
    print("aucun cas n'est vérifié")
\end{verbatim}

\subsection{Exercice}
Dans le programme suivant, quel bloc d'instructions va s'exécuter? Préciser la valeur affichée sur la console.
\begin{verbatim}{python,interactive}
y = 3
if y <= -1:
    print(3 * y + 5)
elif y <= 3 :
    print(y + 4)
else:
    print(y**2 - 1)
\end{verbatim}

\subsection{Exercice}
Un zoo pratique les tarifs suivants.\\
\begin{itemize}
    \item Les enfants jusqu'à 16 ans révolus payent 15 francs.
    \item Les jeunes entre 16 et 20 payent 22 francs.
    \item Les adules à partir de 21 ans payent 28 francs.
\end{itemize}
Écrire un programme qui demande l'âge d'une personne et affiche le prix à payer.
\begin{solution}
\begin{verbatim}{python,interactive}
age = int(input("Quel est votre âge? "))
if age < 16:
    prix = 15
elif age <= 20:
    prix = 22
else:
    prix = 28
print("Pour une personne de " + str(age) + " ans, le prix à payer est de " + str(prix) + " francs.")
\end{verbatim}
\end{solution}

\subsection{Exercice}
Une salle de trampoline pratique les tarifs suivants pour deux personnes.\\
\begin{itemize}
    \item Si les deux personnes sont mineures, elle payent chacune 7 francs.
    \item Si l'une seulement est mineure, elles payent un tarif de groupe se 15 francs.
    \item Si les deux personnes sont majeures, elles payent au total 28 francs.
\end{itemize}
Écrire un programme qui demande l'âge des deux personnes et affiche le prix total à payer.
\begin{solution}
\begin{verbatim}{python,interactive}
age1 = int(input("Quel est l'âge de la première personne? "))
age2 = int(input("Quel est l'âge de la deuxième personne? "))
if (age1 < 18) and (age2 < 18):
    prix = 2 * 7
elif age1 < 18 or age2 < 18:
    prix = 15
else:
    prix = 28
print("Le prix total à payer est de " + str(prix) + " francs.")
\end{verbatim}
\end{solution}

\subsection{Exercice}
Julien souhaite s'inscrire à des séances d'équitation.\\
Le club propose deux types de tarification:
\begin{itemize}
    \item Tarif A: Avec un abonnement annuel de 185 francs, la séance coûte 11 francs.
    \item Tarif B: Sans abonnement, la séance coûte 17 francs.
\end{itemize}
Écrire un programme qui demande à Julien le nombre de séances qu'il voudrait suivre pendant l'année et afficher le tarif le plus avantageux dans ce cas.
\begin{solution}
\begin{verbatim}{python,interactive}
n = int(input("Nombre de séances: "))
prixA = 11 *n + 185
prixB = 17 * n
if prixA < prixB:
    print("Le tarif A est le plus avantageux.")
elif prixB < prixA:
    print("Le tarif B est le plus avantageux.")
else:
    print("Les deux tarifs sont équivalents.")
\end{verbatim}
\end{solution}

\subsection{Exercice}
Écrire un programme qui demande 3 nombres à l'utilisateur et lui retourne le maximum (le plus grand).
\begin{solution}
\begin{verbatim}{python,interactive}
n1 = int(input("Choisir un premier nombre: "))
n2 = int(input("Choisir un deuxième nombre: "))
n3 = int(input("Choisir un troisième nombre: "))
max = n1
if n2 > max:
    max = n2
if n3 > max:
    max = n3
print("Le plus grand nombre est " + str(max))
\end{verbatim}
\end{solution}

\subsection{Exercice}
Écrire un programme qui demande un nombre à l'utilisateur et vérifie s'il est pair ou impair.
\begin{solution}
\begin{verbatim}{python,interactive}
n = int(input("Choisir un nombre: "))
if n % 2 == 0:
    print(str(n) + " est pair.")
else:
    print(str(n) + " est impair.")
\end{verbatim}
\end{solution}

\subsection{Exercice}
Écrire un programme qui demande un nombre à l'utilisateur et vérifie s'il est divisible par 3 et 13.
\begin{solution}
\begin{verbatim}{python,interactive}
n = int(input("Choisir un nombre: "))
if n % 3 == 0 and n % 13 == 0:
    print(str(n) + " est divisible par 3 et 13.")
elif n % 3 == 0:
    print(str(n) + " est divisible par 3.")
elif n % 13 == 0:
    print(str(n) + " est divisible par 13.")
else:
    print(str(n) + " n'est divisible ni par 3 et ni par 13.")
\end{verbatim}
\end{solution}

\section{Boucles}

\subsection{Exercice}
Tu as à nouveau reçu une punition et tu dois écrire "J'aime l'informatique" 100 fois.
\begin{enumerate}
\item Écrire un programme qui permet de faire cela en utilisant une boucle.
\begin{solution}
\begin{verbatim}{python,interactive}
for i in range(100):
    print("J'aime l'informatique")
\end{verbatim}
\end{solution}
\item Tu dois en plus numéroter les lignes
\begin{solution}
\begin{verbatim}{python,interactive}
for i in range(100):
    print(str(i + 1) + ". J'aime l'informatique")
\end{verbatim}
\end{solution}
\item À la fin, tu dois encore afficher: Ouf, j'ai enfin terminé!
\begin{solution}
\begin{verbatim}{python,interactive}
for i in range(100):
    print(str(i + 1) + ". J'aime l'informatique")
print("Ouf, j'ai enfin terminé!")
\end{verbatim}
\end{solution}
\end{enumerate}

\subsection{Exercice}
Écrire un programme qui demande un nombre à l'utilisteur et écrit la table de multiplication de ce nombre:\\
\begin{solution}
\begin{verbatim}{python,interactive}
n = int(input("Choisir un nombre. "))
for i in range(1, 13):
  print(i * n)
\end{verbatim}
\end{solution}

\vfill
\renewcommand{\refname}{Références}
\begin{thebibliography}{9}
\bibitem{py} O.DEFOREST, François: \emph{Python pour le lycée}
\end{thebibliography}

\end{document}
