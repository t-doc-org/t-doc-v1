\documentclass[a4paper,11pt]{article}
\usepackage{commonpackages}

\usepackage{xcolor}
\definecolor{mybackcolor}{rgb}{0.98,0.98,0.98}

\begin{document}
\title{Exercices avec Python}
\date{}
\maketitle

\section{Exercice}
Tu as reçu une punition et tu dois écrire "J'aime l'informatique" 100 fois.
Écrire un progamme qui permet de faire cela en une seule ligne.
\begin{solution}
\begin{verbatim}{python,interactive}
print("J'aime l'informatique! \n" * 100)
\end{verbatim}
\end{solution}


\section{Exercice}
Écrire un programme qui demande son prénom à l'utilisateur et qui répond:\\
Bonjour \textbf{...}\\
\begin{solution}
\begin{verbatim}{python,interactive}
nom = input("Quel est ton prénom? ")
print("Bonjour " + nom)
\end{verbatim}
\end{solution}

\section{Exercice}
Écrire un programme qui demande un nombre à l'utilisateur et qui retourne le carré de ce nombre:\\
\begin{solution}
\begin{verbatim}{python,interactive}
n = int(input("Choisir un nombre: "))
print(n**2)
\end{verbatim}
\end{solution}

\section{Exercice}
Écrire un programme qui son âge à l'utilisteur et retourne s'il est mineur ou majeur:\\
\begin{solution}
\begin{verbatim}{python,interactive}
age = int(input("Quel est votre age? "))
if (age < 18):
  print("Tu es mineur.")
else:
  print("Tu es majeur.")
\end{verbatim}
\end{solution}

\section{Exercice}
Écrire un programme qui demande un nombre à l'utilisteur et écrit la table de multiplication de ce nombre:\\
\begin{solution}
\begin{verbatim}{python,interactive}
n = int(input("Choisir un nombre. "))
for i in range(1, 13):
  print(i * n)
\end{verbatim}
\end{solution}




\end{document}
