\documentclass[a4paper,11pt]{article}
\usepackage{commonpackages}

\begin{document}
\section{Équations du deuxième degré}
\subsection{théorie}
L'expression sous la racine $$\Delta= \mathbf{b^2-4ac}$$ est appelée le \textbf{discriminant} et détermine le nombre de solutions d'une équation quadratique:\\
$\begin{array}{|l | l |}
\hline
\Delta>0 & \text{L'équation a \textbf{deux solutions} réelles} \\
& \text{Le polynôme factorisé est de la forme:}  a(x - x_{1})(x - x_{2}) \\
\hline
\Delta=0 & \text{L'équation a \textbf{une solution} réelle (solution double)}\\
& \text{Le polynôme factorisé est de la forme:}  a(x - x_{1})^2\\
\hline
\Delta<0 & \text{L'équation \textbf{n'a pas de solution} réelle}\\
& \text{Le polynôme ne peut pas être factorisé.}\\
\hline
\end{array}$\par

\subsection{vidéo}
Ici j'ajoute un lien vers une vidéo youtube:\par
\video[100]{tc9wvbYuZts}

\subsection{geogebra}
Ici j'ajoute un lien vers un document geogebra:\par
\geogebra[100]{cpmbrfxg}

\end{document}
