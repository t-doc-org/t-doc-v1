\documentclass[a4paper,11pt]{article}
\usepackage{commonpackages}

\begin{document}
\title{Python}
\date{}
\maketitle


\section{Python}
Ici j'ajoute du code python directement dans le document:\par

\begin{code}{python}
print("Hello World!"*10)
\end{code}


\begin{code}[interactive]{python}
from math import *
for i in range(4):
    for j in range(5):
        print(i, sqrt(j))
\end{code}

\begin{code}[interactive_turtle]{python}
from turtle import *

for i in range(4):
    forward(20)
    penup()
    forward(20)
    pendown()
\end{code}


\subsection{Exercice}
Tu as à nouveau reçu une punition et tu dois écrire "J'aime l'informatique" 100 fois.
\begin{enumerate}
\item Écrire un programme qui permet de faire cela en utilisant une boucle.
\begin{solution}
\begin{code}[interactive]{python}
for i in range(100):
    print("J'aime l'informatique")
\end{code}
\end{solution}
\item Tu dois en plus numéroter les lignes
\begin{solution}
\begin{code}[interactive]{python}
for i in range(100):
    print(str(i + 1) + ". J'aime l'informatique")
\end{code}
\end{solution}
\item À la fin, tu dois encore afficher: Ouf, j'ai enfin terminé!
\begin{solution}
\begin{code}[interactive]{python}
for i in range(100):
    print(str(i + 1) + ". J'aime l'informatique")
print("Ouf, j'ai enfin terminé!")
\end{code}
\end{solution}
\end{enumerate}


\begin{code}{javascript}
import hljs from 'https://cdnjs.cloudflare.com/ajax/libs/highlight.js/11.9.0/es/highlight.min.js';

document.addEventListener('DOMContentLoaded', async (event) => {
    // Add the highlight.js stylesheet.
    const css = document.createElement('link');
    css.rel = 'stylesheet'
    css.href = 'https://cdnjs.cloudflare.com/ajax/libs/highlight.js' +
               '/11.9.0/styles/stackoverflow-light.min.css';
    css.async = true;
    document.head.appendChild(css);

    // Add hljs as a global variable, because highlightjs-line-numbers.js
    // patches it on import, and expects it to be there. We need to use a
    // dynamic import to ensure it happens after setting the global.
    window.hljs = hljs;
    await import('https://cdn.jsdelivr.net/npm/highlightjs-line-numbers.js' +
                 '@2.8.0/dist/highlightjs-line-numbers.min.js');

    // Highlight all verbatim sections.
    document.querySelectorAll('pre.verbatim').forEach((el) => {
        // Extract an optional language identifier prefix of the form
        // "{python}".
        const text = el.innerHTML;
        const re = /^\{([a-zA-Z0-9_-]+)\}( *\n)?/;
        const m = re.exec(text);
        if (m) {  // Found, remove the prefix and add the relevant CSS class
            const lang = m[1];
            el.innerHTML = text.substr(m[0].length);
            el.classList.add(`language-${lang}`);
        }
        hljs.highlightElement(el);
        hljs.lineNumbersBlock(el);
    });
});
\end{code}

\begin{code}{html}
<ul>
  <li>Tomates</li>
  <li>Courgettes</li>
  <li>...</li>
</ul>
\end{code}

\end{document}
